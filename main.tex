\documentclass{article}
\usepackage[left=1in,right=1in]{geometry}
\usepackage{subfiles}
\usepackage{amsmath, amssymb, stmaryrd, verbatim} % math symbols
\usepackage{amsthm} % thm environment
\usepackage{mdframed} % Customizable Boxes
\usepackage{hyperref,nameref,cleveref,enumitem} % for references, hyperlinks
\usepackage[dvipsnames]{xcolor} % Fancy Colours
\usepackage{mathrsfs} % Fancy font
\usepackage{bbm} % mathbb numerals
\usepackage[bbgreekl]{mathbbol} % for bb Delta
\usepackage{tikz, tikz-cd, float} % Commutative Diagrams
\usetikzlibrary{decorations.pathmorphing} % for squiggly arrows in tikzcd
\usepackage{perpage}
\usepackage{parskip} % So that paragraphs look nice
\usepackage{ifthen,xargs} % For defining better commands
\usepackage[T1]{fontenc}
\usepackage[utf8]{inputenc}
\usepackage{tgpagella}
\usepackage{cancel}

% Bibliography
% \usepackage{url}
\usepackage[backend=biber,
            isbn=false,
            doi=false,
            giveninits=true,
            style=alphabetic,
            useprefix=true,
            maxcitenames=4,
            maxbibnames=4,
            sorting=nyt,
            citestyle=alphabetic]{biblatex}

% Shortcuts

% % Local to this project
\renewcommand\labelitemi{--} % Makes itemize use dashes instead of bullets
\newcommand{\bbrkt}[1]{\llbracket #1 \rrbracket}
\newcommand{\IND}{\mathrm{Ind}}
\newcommand{\AFF}{\mathrm{Aff}}
\newcommand{\FSCH}{\mathrm{fSch}}
\newcommand{\SCH}{\mathrm{Sch}}
\DeclareMathOperator{\SPEC}{Spec}
\DeclareMathOperator{\QCOH}{QCoh}
\DeclareMathOperator{\INDCOH}{IndCoh}
\newcommand{\DR}{\mathrm{dR}}
\newcommand{\SYM}{\mathrm{Sym}}
\newcommand{\HOM}{\underline{\mathrm{Hom}}}
\newcommand{\CALG}{\mathrm{cAlg}}
\DeclareMathOperator{\GL}{GL}
\DeclareMathOperator{\LIE}{Lie}
\newcommand{\AD}{\mathrm{Ad}}
\DeclareMathOperator{\MAX}{Max}
\DeclareMathOperator{\FROB}{Frob}
\newcommand{\SEP}{\mathrm{sep}}
\newcommand{\LOC}{\mathrm{Loc}}
\DeclareMathOperator{\PERV}{Perv}
\newcommand{\RED}{\mathrm{red}}
\newcommand{\ZAR}{\mathrm{Zar}}
\newcommand{\FPPF}{\mathrm{fppf}}
\newcommand{\ET}{\text{ét}}
\DeclareMathOperator{\SPF}{Spf}
\newcommand{\CTS}{\mathrm{cts}}
\newcommand{\INF}{\mathrm{Inf}}
\DeclareMathOperator{\STK}{Stk}
\DeclareMathOperator{\PSTK}{PStk}
\DeclareMathOperator{\SSET}{sSet}
\DeclareMathOperator{\BUN}{Bun}
\DeclareMathOperator{\PSH}{PSh}
\DeclareMathOperator{\SH}{Sh}
\newcommand{\FPQC}{\mathrm{fpqc}}
\DeclareMathOperator{\SPH}{Sph}
\newcommand{\FT}{\mathrm{ft}}
\DeclareMathOperator{\CRYS}{Crys}
\DeclareMathOperator{\GR}{Gr}
\DeclareMathOperator{\REP}{Rep}
\newcommand{\HITCH}{\mathrm{Hitch}}
\newcommand{\CRIT}{\mathrm{crit}}
\DeclareMathOperator{\LOCSYS}{LocSys}
\newcommand{\GRPD}{\mathrm{Grpd}}
\DeclareMathOperator{\MAP}{Map}
\DeclareMathOperator{\FUN}{Fun}
\newcommand{\HECKE}{\mathrm{Hecke}}
\newcommand{\TRIV}{\mathrm{Triv}}
\newcommand{\REGLUE}{\mathrm{ReGlue}}
\newcommand{\LVL}{\mathrm{lvl}}
\newcommand{\PT}{\mathrm{pt}}
\DeclareMathOperator{\COH}{Coh}
\DeclareMathSymbol\bbDe  \mathord{bbold}{"01} % blackboard Delta
\newcommand{\CL}{\mathrm{cl}}
\newcommand{\OPER}{\operatorname{Oper}}
\DeclareMathOperator{\PERF}{Perf}

% % Misc
\newcommand{\brkt}[1]{\left(#1\right)}
\newcommand{\sqbrkt}[1]{\left[#1\right]}
\newcommand{\dash}{\text{-}}

% % Logic
\renewcommand{\implies}{\Rightarrow}
\renewcommand{\iff}{\Leftrightarrow}
\newcommand{\limplies}{\Leftarrow}
\newcommand{\NOT}{\neg\,}
\newcommand{\AND}{\, \land \,}
\newcommand{\OR}{\, \lor \,}
\newenvironment{forward}{($\implies$)}{}
\newenvironment{backward}{($\limplies$)}{}

% % Sets
\DeclareMathOperator{\supp}{supp}
\newcommand{\set}[1]{\left\{#1\right\}}
\newcommand{\st}{\,|\,}
\newcommand{\minus}{\setminus}
\newcommand{\subs}{\subseteq}
\newcommand{\ssubs}{\subsetneq}
\newcommand{\sups}{\supseteq}
\newcommand{\ssups}{\supset}
\DeclareMathOperator{\im}{Im}
\newcommand{\nothing}{\varnothing}
\DeclareMathOperator{\join}{\sqcup}
\DeclareMathOperator{\meet}{\sqcap}

% % Greek 
\newcommand{\al}{\alpha}
\newcommand{\be}{\beta}
\newcommand{\ga}{\gamma}
\newcommand{\de}{\delta}
\newcommand{\ep}{\varepsilon}
\newcommand{\ph}{\varphi}
\newcommand{\io}{\iota}
\newcommand{\ka}{\kappa}
\newcommand{\la}{\lambda}
\newcommand{\om}{\omega}
\newcommand{\si}{\sigma}

\newcommand{\Ga}{\Gamma}
\newcommand{\De}{\Delta}
\newcommand{\Th}{\Theta}
\newcommand{\La}{\Lambda}
\newcommand{\Si}{\Sigma}
\newcommand{\Om}{\Omega}

% % Mathbb
\newcommand{\bA}{\mathbb{A}}
\newcommand{\bB}{\mathbb{B}}
\newcommand{\bC}{\mathbb{C}}
\newcommand{\bD}{\mathbb{D}}
\newcommand{\bE}{\mathbb{E}}
\newcommand{\bF}{\mathbb{F}}
\newcommand{\bG}{\mathbb{G}}
\newcommand{\bH}{\mathbb{H}}
\newcommand{\bI}{\mathbb{I}}
\newcommand{\bJ}{\mathbb{J}}
\newcommand{\bK}{\mathbb{K}}
\newcommand{\bL}{\mathbb{L}}
\newcommand{\bM}{\mathbb{M}}
\newcommand{\bN}{\mathbb{N}}
\newcommand{\bO}{\mathbb{O}}
\newcommand{\bP}{\mathbb{P}}
\newcommand{\bQ}{\mathbb{Q}}
\newcommand{\bR}{\mathbb{R}}
\newcommand{\bS}{\mathbb{S}}
\newcommand{\bT}{\mathbb{T}}
\newcommand{\bU}{\mathbb{U}}
\newcommand{\bV}{\mathbb{V}}
\newcommand{\bW}{\mathbb{W}}
\newcommand{\bX}{\mathbb{X}}
\newcommand{\bY}{\mathbb{Y}}
\newcommand{\bZ}{\mathbb{Z}}

% % Mathcal
\newcommand{\cA}{\mathcal{A}}
\newcommand{\cB}{\mathcal{B}}
\newcommand{\cC}{\mathcal{C}}
\newcommand{\cD}{\mathcal{D}}
\newcommand{\cE}{\mathcal{E}}
\newcommand{\cF}{\mathcal{F}}
\newcommand{\cG}{\mathcal{G}}
\newcommand{\cH}{\mathcal{H}}
\newcommand{\cI}{\mathcal{I}}
\newcommand{\cJ}{\mathcal{J}}
\newcommand{\cK}{\mathcal{K}}
\newcommand{\cL}{\mathcal{L}}
\newcommand{\cM}{\mathcal{M}}
\newcommand{\cN}{\mathcal{N}}
\newcommand{\cO}{\mathcal{O}}
\newcommand{\cP}{\mathcal{P}}
\newcommand{\cQ}{\mathcal{Q}}
\newcommand{\cR}{\mathcal{R}}
\newcommand{\cS}{\mathcal{S}}
\newcommand{\cT}{\mathcal{T}}
\newcommand{\cU}{\mathcal{U}}
\newcommand{\cV}{\mathcal{V}}
\newcommand{\cW}{\mathcal{W}}
\newcommand{\cX}{\mathcal{X}}
\newcommand{\cY}{\mathcal{Y}}
\newcommand{\cZ}{\mathcal{Z}}

% % Mathfrak
\newcommand{\f}[1]{\mathfrak{#1}}

% % Mathrsfs
\newcommand{\s}[1]{\mathscr{#1}}

% % Category Theory
\DeclareMathOperator{\obj}{Obj}
\DeclareMathOperator{\END}{End}
\DeclareMathOperator{\AUT}{Aut}
\newcommand{\CAT}{\mathbf{Cat}}
\newcommand{\SET}{\mathbf{Set}}
\newcommand{\TOP}{\mathbf{Top}}
\newcommand{\MON}{\mathbf{Mon}}
\newcommand{\GRP}{\mathbf{Grp}}
\newcommand{\AB}{\mathbf{Ab}}
\newcommand{\RING}{\mathbf{Ring}}
\newcommand{\CRING}{\mathbf{CRing}}
\newcommand{\MOD}{\mathbf{Mod}}
\newcommand{\VEC}{\mathbf{Vec}}
\newcommand{\ALG}{\mathbf{Alg}}
\newcommand{\ORD}{\mathbf{Ord}}
\newcommand{\POSET}{\mathbf{PoSet}}
\newcommand{\id}[1]{\mathbbm{1}_{#1}}
\newcommand{\map}[2]{\yrightarrow[#2][2.5pt]{#1}[-1pt]}
\newcommand{\iso}[1][]{\cong_{#1}}
\newcommand{\OP}{\mathrm{op}}
\newcommand{\darrow}{\downarrow}
\newcommand{\LIM}{\varprojlim}
\newcommand{\COLIM}{\varinjlim}
\DeclareMathOperator{\coker}{coker}
\newcommand{\fall}[2]{\downarrow_{#2}^{#1}}
\newcommand{\lift}[2]{\uparrow_{#1}^{#2}}

% % Algebra
\newcommand{\nsub}{\trianglelefteq}
\newcommand{\inv}{{-1}}
\newcommand{\dvd}{\,|\,}
\DeclareMathOperator{\ev}{ev}

% % Analysis
\newcommand{\abs}[1]{\left\vert #1 \right\vert}
\newcommand{\norm}[1]{\left\Vert #1 \right\Vert}
\renewcommand{\bar}[1]{\overline{#1}}
\newcommand{\<}{\langle}
\renewcommand{\>}{\rangle}
\renewcommand{\hat}[1]{\widehat{#1}}
\renewcommand{\check}[1]{\widecheck{#1}}
\newcommand{\dsum}[2]{\sum_{#1}^{#2}}
\newcommand{\dprod}[2]{\prod_{#1}^{#2}}
\newcommand{\del}[2]{\frac{\partial#1}{\partial#2}}
\newcommand{\res}[2]{{% we make the whole thing an ordinary symbol
  \left.\kern-\nulldelimiterspace % automatically resize the bar with \right
  #1 % the function
  %\vphantom{\big|} % pretend it's a little taller at normal size
  \right|_{#2} % this is the delimiter
  }}

% % Galois
\DeclareMathOperator{\GAL}{Gal}
\DeclareMathOperator{\ORB}{Orb}
\DeclareMathOperator{\STAB}{Stab}
\newcommand{\emb}[3]{\mathrm{Emb}_{#1}(#2, #3)}
\newcommand{\Char}[1]{\mathrm{Char}#1}

%% code from mathabx.sty and mathabx.dcl to get some symbols from mathabx
\DeclareFontFamily{U}{mathx}{\hyphenchar\font45}
\DeclareFontShape{U}{mathx}{m}{n}{
      <5> <6> <7> <8> <9> <10>
      <10.95> <12> <14.4> <17.28> <20.74> <24.88>
      mathx10
      }{}
\DeclareSymbolFont{mathx}{U}{mathx}{m}{n}
\DeclareFontSubstitution{U}{mathx}{m}{n}
\DeclareMathAccent{\widecheck}{0}{mathx}{"71}

% Arrows with text above and below with adjustable displacement
% (Stolen from Stackexchange)
\newcommandx{\yaHelper}[2][1=\empty]{
\ifthenelse{\equal{#1}{\empty}}
  % no offset
  { \ensuremath{ \scriptstyle{ #2 } } } 
  % with offset
  { \raisebox{ #1 }[0pt][0pt]{ \ensuremath{ \scriptstyle{ #2 } } } }  
}

\newcommandx{\yrightarrow}[4][1=\empty, 2=\empty, 4=\empty, usedefault=@]{
  \ifthenelse{\equal{#2}{\empty}}
  % there's no text below
  { \xrightarrow{ \protect{ \yaHelper[ #4 ]{ #3 } } } } 
  % there's text below
  {
    \xrightarrow[ \protect{ \yaHelper[ #2 ]{ #1 } } ]
    { \protect{ \yaHelper[ #4 ]{ #3 } } } 
  } 
}

% xcolor
\definecolor{darkgrey}{gray}{0.10}
\definecolor{lightgrey}{gray}{0.30}
\definecolor{slightgrey}{gray}{0.80}
\definecolor{softblue}{RGB}{30,100,200}

% hyperref
\hypersetup{
      colorlinks = true,
      linkcolor = {softblue},
      citecolor = {blue}
}

\newcommand{\link}[1]{\hypertarget{#1}{}}
\newcommand{\linkto}[2]{\hyperlink{#1}{#2}}

% Perpage
\MakePerPage{footnote}

% Theorems

% % custom theoremstyles
\newtheoremstyle{definitionstyle}
{5pt}% above thm
{0pt}% below thm
{}% body font
{}% space to indent
{\bf}% head font
{\vspace{1mm}}% punctuation between head and body
{\newline}% space after head
{\thmname{#1}\thmnote{\,\,--\,\,#3}}

\newtheoremstyle{exercisestyle}%
{5pt}% above thm
{0pt}% below thm
{\it}% body font
{}% space to indent
{\it}% head font
{.}% punctuation between head and body
{ }% space after head
{\thmname{#1}\thmnote{ (#3)}}

\newtheoremstyle{examplestyle}%
{5pt}% above thm
{0pt}% below thm
{\it}% body font
{}% space to indent
{\it}% head font
{.}% punctuation between head and body
{\newline}% space after head
{\thmname{#1}\thmnote{ (#3)}}

\newtheoremstyle{remarkstyle}%
{5pt}% above thm
{0pt}% below thm
{}% body font
{}% space to indent
{\it}% head font
{.}% punctuation between head and body
{ }% space after head
{\thmname{#1}\thmnote{\,\,--\,\,#3}}

\newtheoremstyle{questionstyle}%
{5pt}% above thm
{0pt}% below thm
{}% body font
{}% space to indent
{\it}% head font
{?}% punctuation between head and body
{ }% space after head
{\thmname{#1}\thmnote{\,\,--\,\,#3}}

% Custom Environments

% % Theorem environments

\theoremstyle{definitionstyle}
\newmdtheoremenv[
    linewidth = 2pt,
    leftmargin = 0pt,
    rightmargin = 0pt,
    linecolor = darkgrey,
    topline = false,
    bottomline = false,
    rightline = false,
    footnoteinside = true
]{dfn}{Definition}
\newmdtheoremenv[
    linewidth = 2 pt,
    leftmargin = 0pt,
    rightmargin = 0pt,
    linecolor = darkgrey,
    topline = false,
    bottomline = false,
    rightline = false,
    footnoteinside = true
]{prop}{Proposition}
\newmdtheoremenv[
    linewidth = 2 pt,
    leftmargin = 0pt,
    rightmargin = 0pt,
    linecolor = darkgrey,
    topline = false,
    bottomline = false,
    rightline = false,
    footnoteinside = true
]{cor}{Corollary}

\theoremstyle{exercisestyle}
\newmdtheoremenv[
    linewidth = 0.7 pt,
    leftmargin = 20pt,
    rightmargin = 0pt,
    linecolor = darkgrey,
    topline = false,
    bottomline = false,
    rightline = false,
    footnoteinside = true
]{ex}{Exercise}
\newmdtheoremenv[
    linewidth = 0.7 pt,
    leftmargin = 20pt,
    rightmargin = 0pt,
    linecolor = darkgrey,
    topline = false,
    bottomline = false,
    rightline = false,
    footnoteinside = true
]{lem}{Lemma}

\theoremstyle{examplestyle}
\newmdtheoremenv[
    linewidth = 0.7 pt,
    leftmargin = 0pt,
    rightmargin = 0pt,
    linecolor = darkgrey,
    topline = false,
    bottomline = false,
    rightline = false,
    footnoteinside = true
]{eg}{Example}
\newmdtheoremenv[
    linewidth = 0.7 pt,
    leftmargin = 0pt,
    rightmargin = 0pt,
    linecolor = darkgrey,
    topline = false,
    bottomline = false,
    rightline = false,
    footnoteinside = true
]{ceg}{Counter Example}

\theoremstyle{remarkstyle}
\newtheorem{rmk}{Remark}

\theoremstyle{questionstyle}
\newtheorem{question}{Question}

\newenvironment{proof1}{
  \begin{proof}\renewcommand\qedsymbol{$\blacksquare$}
}{
  \end{proof}
} % Proofs ending with black qedsymbol 

% % tikzcd diagram 
\newenvironment{cd}{
    \begin{figure}[H]
    \centering
    \begin{tikzcd}
}{
    \end{tikzcd}
    \end{figure}
}

% tikzcd
% % Substituting symbols for arrows in tikz comm-diagrams.
\tikzset{
  symbol/.style={
    draw=none,
    every to/.append style={
      edge node={node [sloped, allow upside down, auto=false]{$#1$}}}
  }
}

\addbibresource{mybib.bib}

\begin{document}

\title{Week 4 : Hecke actions}

\author{Ken Lee}
\date{Summer 2023}
\maketitle

The goal of this talk is to give a more detailed introduction
to the spherical Hecke category in preparation for the
three talks on geometric Satake.

Brief recollection of week 1 talk : 
\begin{enumerate}
  \item Throughout, we fixed a smooth projective curve $X$ over
  a base field $k$ with characteristic zero.
  For simplicity, one should assume it to be $\bC$.
  We also fix a reductive group $G$ over $k$.
  \item $\BUN_G$ is the algebro-geometric version of the
  double coset of adeles in 
  unramified Langlands of function fields over a finite field.
  \item Hecke eigenforms, 
  which are certain functions on the double coset space of the adeles,
  can be constructed from decategorifying 
  perverse sheaves on $\BUN_G$.
  \item Over characteristic zero, perverse sheaves and $D$-modules are
  equivalent by the Riemann--Hilbert correspondence.
  Categorical geometric Langlands takes this as the starting point.
  \item affine Grassmannians are the algebro-geometric versions of
  the homogeneous space $G((t)) / G\bbrkt{t}$
  and hence lead to the geometric version of spherical Hecke algebras $\SPH_G$.
  \item Given a closed point $x$ of the curve,
  there is an action of $\SPH_G$ on $D\MOD(\BUN_G)$ called the
  automorphic Hecke action.
  \item Given a $G^L$ local system $\si$ on $X$,
  there is an action of $\REP_k G^L$ on $D\MOD(\BUN_G)$
  called the Galois Hecke action.
  \item Geometric Satake gives $\SPH_G \simeq \REP_k G^L$
  as tensor categories.
  \item The automorphic Hecke action across all points can be
  bundled to together, giving two actions of $\REP_k G^L$
  on $D\MOD(\BUN_G)$.
  \item modulo details about associativity,
  a Hecke eigensheaf is a $D$-module on $\BUN_G$
  ``such that'' the two actions of $\REP_k G^L$ coincide.
  \footnote{
    We put ``such that'' in quotation marks because
    this is a structure, not a property.
  }
\end{enumerate}
This talk focuses on giving more details on
(5), (6), (9).

Setup some notation : 
\begin{itemize}
  \item $\AFF_k$ denotes the opposite of the category of $k$-algebras.
  \item For an affine scheme $S = \SPEC A$,
  $D^\wedge_S , D_S , D_S^\circ$ will denote
  $\SPF A\bbrkt{t} , \SPEC A\bbrkt{t} , \SPEC A((t))$.
  These are the relative formal disk, disk, punctured disk over $S$.
\end{itemize}

\tableofcontents  

\section{Arc groups, Loop groups and Affine Grassmannians}

\begin{dfn}[Arc and loop group]
  
  The \emph{arc group of $G$} is defined as the functor on $\AFF_k$
  \[
    L^+G(S) := G(D_S)
  \]
  and the \emph{loop group of $G$} is defined as the functor
  \[
    LG(S) := G(D_S^\circ)
  \]
\end{dfn}

Ideally, $L^+G$ should be thought of as
the mapping space $\HOM(D , G)$ from the disk $D$ to $G$,
and similarly $LG$ should be $\HOM(D^\circ , G)$.
The issue is $A \otimes_k k\bbrkt{t} \not\simeq A\bbrkt{t}$ so
this is not literally true.
This can be fixed by using the formal disk $D^\wedge := \SPF k\bbrkt{t}$ instead
since $\SPEC A \times \SPF k\bbrkt{t} = \SPF A\bbrkt{t}$
and $G$ is affine.
However, this fix does not work for the loop group $LG$ because
``one cannot $\SPF$ the topological ring $A((t))$''.

Restricting along $D_S^\circ \to D_S$ defines a
morphism $L^+G \to LG$ of group functors.
One can show this is a closed embedding.
\begin{lem}
  The morphism $L^+G \to LG$ is a closed embedding.
  \begin{proof1}
    Since $G$ is affine,
    there exists a closed embedding $i : G \to \bA^N$ for some finite $N$.
    Let $s : S \to LG$ be a point of the loop group.
    Using $i$, we see that $s$ is given by $N$-coordinates
    $(s_1 , \dots , s_N) \in \bA^N(D_S^\circ)$.
    The fiber product $L^+ G \times_{LG} S$ is computed as
    the vanishing locus of the finitely many polar terms
    of $s_1 , \dots, s_N$.
  \end{proof1}
\end{lem}

Here's the answer to the question of what kind of objects are
the arc and loop group.
\begin{prop}
  
  The arc group $L^+G$ is a scheme and the loop group $LG$
  is an ind-affine scheme.
\end{prop}
\begin{proof}
  
  We define intermediate spaces : 
  the space of $n$-jets $L^n G$ is defined as a functor by
  \[
    L^n G := \HOM(\SPEC k[t] / (t^{n+1}) , G)
  \]
  Note that $n = 1$ recovers the tangent bundle of $G$.
  It is not hard to show that in the case of $G = \bA^1$ that
  $L^n G$ is affine.
  Since $G$ is affine finite type, it can be put into a 
  cartesian square 
  \begin{cd}
    G & {\bA^N} \\
    0 & {\bA^r}
    \arrow[from=1-2, to=2-2]
    \arrow[from=2-1, to=2-2]
    \arrow[from=1-1, to=2-1]
    \arrow[from=1-1, to=1-2]
    \arrow["\lrcorner"{anchor=center, pos=0.125}, draw=none, from=1-1, to=2-2]
  \end{cd}
  Since $L^n$ is right adjoint to $\_ \times \SPEC k[t] / (t^{n+1})$,
  is preserves limits so $L^n G$ is affine.
  Finally, $L^+ G \simeq \LIM_{n} L^n G$ is hence also affine.

  Now for the loop group.
  It is \emph{not} true that $LG = \HOM(D^\circ , G)$,
  but the formula 
  \[
    (LG)(S) = G(D_S^\circ) \overset{\text{Yoneda}}{\simeq} \mathrm{Hom}(D_S^\circ , G)
  \]
  still shows that $L$ commutes with limits.
  By the same argument as above, 
  we reduce to $G = \bA^1$.
  One then computes 
  \[
    \bA^1((t))(A) = \bA^1(A((t))) = A((t))
    \simeq \COLIM_n A\bbrkt{t} = (\COLIM_n \bA^1\bbrkt{t}) (A)
  \]
  where the system we are taking colimits is 
  \[
    \bA^1\bbrkt{t} \map{t}{} \bA^1\bbrkt{t} \map{t}{} \cdots
  \]
  Concretely, $\bA^1\bbrkt{t} \simeq \bA^\bN$
  and the above transition maps shift the sequence of coefficients 
  to the right by one.
  One sees from this that each copy $\bA^1\bbrkt{t}$
  is the closed subscheme of the next copy of 
  sequences where the first coefficient is zero.
  We have thus shown $\bA^1((t))$ is a ind-affine scheme.
\end{proof}

\begin{dfn}[Affine Grassmannian for general $G$]
  
  Let $G$ be an algebraic group scheme over $k$.
  For an affine $S$,
  define the groupoid $\GR_G(S)$ as follows :
  \begin{itemize}
    \item Objects are $G$-bundles on $D_S$ together with
    a trivialisation $s : \mathrm{Triv} \simeq \res{P}{D_S^\circ}$
    on the punctured disk $D_S^\circ$.
    \item A morphism $(P , s) \to (Q , t)$ is 
    a morphism $\al : P \to Q$ of $G$-bundles on $D_S$
    respecting the trivialisations, i.e.
    $t = \al s$.
  \end{itemize}
  Given a map $T \to S$ of affines,
  there is a functor $\GR_G(S) \to \GR_G(T)$ given by
  pulling back along $D_T \to D_S$.
  The above forms a groupoid valued (pseudo-)functor on $\AFF_k$,
  which we call the \emph{affine Grassmannian of $G$}.

  \footnote{
    Question to self : 
    Why does Gaitsgory bother writing this using Tannaka formalism when
    $G$-bundles over $D_S$ and $D_S^\circ$ are perfectly well defined?
  }

\end{dfn}

Although $\GR_G$ is a-priori valued in groupoids,
one can show that it is in fact valued in discrete groupoids
by the following lemma.
Hence $\GR_G$ is equivalently a set-valued functor
sending affines $S$ to the isomorphism class of
$G$-bundles on $D_S$ equipped with a trivialisation over $D_S^\circ$.

\begin{lem}

  Let $P$ be a $G$-bundle on $D_S$ where $S$ is affine
  and $s : \TRIV \simeq \res{P}{D_S^\circ}$ a trivialisation
  on the punctured disk.
  Suppose $\al : P \to P$ is an (auto)morphism of $G$-bundles on $D_S$
  such that $\al s = s$ over $D_S^\circ$.
  Then $\al$ must be the identity.

  \begin{proof1}
    
    (Example of trivial line bundle) 
    Suppose $G = \GL_1$ and replace $\GL_1$-bundles with line bundles
    and consider the case of $P = \bA^1_{D_S}$.
    The trivialisation $s$ is equivalent to
    an element $\cO(S)((t))^\times$.
    Similarly, $\al$ is equivalent to an element 
    $\al \in \cO(S)\bbrkt{t}^\times$.
    Then we have $\al s = s$ as elements in $\cO(S)((t))^\times$.
    Since $s$ is a unit, $\al = 1$ in $\cO(S)((t))^\times$ and 
    hence in $\cO(S)\bbrkt{t}^\times$.

    (General case) Using the same strategy as in the example,
    we see that we only need to prove two things : 
    \begin{enumerate}
      \item $P$ can be trivialised after $D_{\tilde{S}} \to D_S$
      by some fpqc cover $\tilde{S} \to S$.
      \item After (1), we can assume $P = G \times D_S$.
      We then need to show $G(D_S) \to G(D_S^\circ)$ is injective.
    \end{enumerate}

    For (1), the special fiber $P_0$ over $\set{0}_S \simeq S$
    is also a $G$-bundle, and hence trivialises upon
    passing to some fpqc cover $\tilde{S} \to S$.
    The trivialisation of $P_0$ over 
    $\tilde{S} \simeq \set{0}_{\tilde{S}}$ extends
    to $D_{\tilde{S}}$ by smoothness of $G$.

    For (2), using affinity of $G$ we can realise $G$
    as a closed subscheme of $\bA^N$ for some large $N$.
    WLOG $N = 1$.
    This reduces to showing $\bA^1(D_S) \to \bA^1(D_S^\circ)$ is injective.
    This is precisely the inclusion $\cO(S)\bbrkt{t} \subs \cO(S)((t))$.
    
  \end{proof1}
\end{lem}

% For some reason \cite{Zhu-16} talks about
% $\GR_{\underline{G}}$ for 
% $\underline{G}$ affine group scheme smooth over $D$.
% The usual case is when $\underline{G} = D \times_k G$ for some
% affine group scheme $G$ over $k$.

We now give the description of the affine Grassmannian as
a homogeneous space of the loop group.

\begin{dfn}
  
  Define an action $\GR_G \times LG \to \GR_G$ as follows
  \[
    ((P , s) , g) \mapsto (P , s g)
  \]
\end{dfn}

\begin{prop}
  
  The map $LG \to \GR_G$ sending $g \mapsto (\TRIV , g)$
  is a $L^+G$-bundle.
\end{prop}
\begin{proof}
  
  We need to show the following : 
  \begin{enumerate}
    \item $L^+G$ acts trivially on $\GR_G$
    \item the map $LG \to \GR_G$ is $L^+G$ equivariant
    \item For every point $(P , s) : S \to \GR_G$ of $\GR_G$ with $S$ affine,
    there exists an fpqc cover $\tilde{S} \to S$ such that 
    the pullback of $LG$ to $\tilde{S}$ is isomorphic to 
    the trivial $L^+G$ bundle.
    \begin{cd}
    	LG & {LG \times_{\GR_G} S} & {\tilde{S} \times L^+G} \\
	{\GR_G} & S & {\tilde{S}}
	\arrow[from=1-1, to=2-1]
	\arrow["{(P ,s)}", from=2-2, to=2-1]
	\arrow["{\text{fpqc}}", from=2-3, to=2-2]
	\arrow[from=1-2, to=1-1]
	\arrow[from=1-2, to=2-2]
	\arrow["\lrcorner"{anchor=center, pos=0.125, rotate=-90}, draw=none, from=1-2, to=2-1]
	\arrow[from=1-3, to=1-2]
	\arrow[from=1-3, to=2-3]
	\arrow["\lrcorner"{anchor=center, pos=0.125, rotate=-90}, draw=none, from=1-3, to=2-2]
    \end{cd}
  \end{enumerate}
  Goal (1) and (2) are clear.
  (3) follows from the fact we've already seen : 
  there exists fpqc cover $\tilde{S} \to S$
  such that $D_{\tilde{S}} \to D_S$ trivialises $P$.

\end{proof}

\section{Spherical $D$-modules, Convolution}

[Can skip until...]
I find it useful to make the following auxiliary definitions,
which are not explicitly present in Gaitsgory's notes.

\begin{dfn}
  
  The \emph{stack of $G$-bundles on the disk} is defined as
  by the groupoid-valued functor
  \[
    (\BUN_G D)(S) := \mathrm{Hom}(D_S , BG)
  \]
  And the \emph{stack of $G$-bundles on the punctured disk} is defined 
  by the groupoid-valued functor
  \[
    (\BUN_G D^\circ)(S) := \mathrm{Hom}(D_S^\circ , BG)
  \]
\end{dfn}

Recall that the definition of $\BUN_G X$ is 
the internal hom $\HOM(X , BG)$.
This begs the question whether $\BUN_G D , \BUN_G D^\circ$ are also 
internal homs.
\begin{enumerate}
  \item For $\BUN_G D$ this is true by a combination of
  the Tannakian formalism and the fact that
  for Noetherian $A$ with ideal $I$,
  finitely generated projective modules over $\SPEC A^\wedge_I$ and 
  $\SPF A^\wedge_I$ are equivalent :
  \begin{align*}
    \mathrm{Hom}(D_S , BG) 
    &\simeq \FUN^\otimes_{\text{ex, cts}}(\REP G , \QCOH D_S)
    \simeq \FUN^\otimes_{\text{ex}}(\REP_{\text{f.d.}} G , \PERF D_S) \\
    &\simeq \FUN^\otimes_{\text{ex}}(\REP_{\text{f.d.}} G , \LIM_n \PERF(n \cdot 0_S)) \\
    &\simeq \LIM_n \FUN^\otimes_{\text{ex}}(\REP_{\text{f.d.}} G , \PERF(n \cdot 0_S)) \\
    &\simeq \LIM_n \mathrm{Hom}(n \cdot 0_S , BG)
    \simeq \mathrm{Hom}(\COLIM_n n \cdot 0_S , BG) \\
    &\simeq \mathrm{Hom}(S \times D^\wedge , BG)
    \simeq \HOM(D^\wedge , BG)(S)
  \end{align*}
  \item For $\BUN_G D^\circ$ this is \emph{not} true,
  again essentially because $\SPEC A((t))$ is not computed as the fiber product
  $\SPEC A \times \SPEC k((t))$.
\end{enumerate}

\begin{dfn}
  
  The local Hecke correspondence is defined as the fiber product
  \begin{cd}
    {\BUN_G D} & {\HECKE^\LOC} \\
    {\BUN_G D^\circ} & {\BUN_G D}
    \arrow[from=1-1, to=2-1]
    \arrow[from=2-2, to=2-1]
    \arrow[from=1-2, to=1-1]
    \arrow[from=1-2, to=2-2]
    \arrow["\lrcorner"{anchor=center, pos=0.125, rotate=-90}, draw=none, from=1-2, to=2-1]
  \end{cd}
\end{dfn}

[... here.]
Unraveling the definition,
a map $S \to \HECKE^\LOC$ from an affine is the same as
the data $(P_0 , P_1 , \al)$ where $P_i$ are $G$-bundles on $D_S$
and $\al : \res{P_0}{D_S^\circ} \simeq \res{P_1}{D_S^\circ}$.
This is related to the affine Grassmannian in the following way : 
\begin{prop}
  
  Define $\HECKE^\LOC_\square$ by the cartesian square 
  \begin{cd}
    {\HECKE^\LOC} & {\HECKE^\LOC_\square} \\
    {\BUN_G D \times\BUN_G D} & \PT
    \arrow[from=1-1, to=2-1]
    \arrow["{(\TRIV , \TRIV)}", from=2-2, to=2-1]
    \arrow[from=1-2, to=1-1]
    \arrow[from=1-2, to=2-2]
    \arrow["\lrcorner"{anchor=center, pos=0.125, rotate=-90}, draw=none, from=1-2, to=2-1]
  \end{cd}
  Then \begin{enumerate}
    \item $L^+ G \backslash \HECKE^\LOC_\square / L^+G \simeq \HECKE^\LOC$
    \item $\HECKE^\LOC_\square \simeq LG$ respecting the
    left and right actions of $L^+G$.
    Hence $\HECKE^\LOC \simeq L^+G \backslash \GR_G$
  \end{enumerate}
\end{prop}
\begin{proof}
  (1) $\HECKE^\LOC_\square$ parameterises 
  $(P_0 , P_1 , \al , s_0 : \TRIV \to P_0 , s_1 : \TRIV \to P_1)$
  where $P_i$ are $G$-bundles on the disk.
  Quotienting by $L^+G$ on both sides forgets the trivialisations $s_i$.

  (2) We give mutual inverse functors and omit the check that it works.

  $LG \to \HECKE^\LOC_\square$ takes $g \in LG(S)$ to
  $(\TRIV , \TRIV , g , \id{} , \id{})$.
  
  $\HECKE^\LOC_\square \to LG$ takes an $S$-point $(P_0 , P_1 , \al , s_0 , s_1)$
  to $s_1^{-1} \al s_0 \in LG(S)$.
\end{proof}

We now want to define the spherical Hecke category,
which $D$-module version of the categorified spherical Hecke category.
Ideally, we want the following definition : 
\[
  \SPH_G := D\MOD(\HECKE^\LOC)
\]
The issue is that at this point, we only really know how to do
$D$-modules on varieties.
The idea is to use $\HECKE^\LOC \simeq L^+G \backslash \GR_G$,
to reduce to the case of varieties.
There are two parts to this reduction : 
\begin{enumerate}
  \item $\GR_G$ is ind-projective,
  meaning we can write it as the colimit of projective varieties
  along closed embeddings.
  This will be addressed in next week's talk 
  on the geometry of the affine Grassmannian.
  \item $D$-modules on quotient stacks.
  It will also be shown next week that
  one can present $\GR_G$ as a filtered colimit of
  projective varieties $X_i$ along closed embeddings
  such that each $X_i$ is $L^+G$-invariant and
  the action in fact factors through an algebraic quotient of $L^+G$.
  In preparation for that,
  we give a short exposition of $D$-modules on
  quotient stacks of varieties by algebraic groups in the next section.
\end{enumerate}
% The following will be shown in the next talk on affine Grassmannians.
% \begin{enumerate}
%   \item $\GR_G$ is ind-projective,
%   meaning we can write it as the colimit of projective varieties
%   along closed embeddings.
%   In this situation of ind-projective varieties,
%   there is a way of defining $D\MOD(\GR_G)$ such that
%   for any presentation $\GR_G = \COLIM_i X_i$ as
%   a colimit of projective varieties along closed embeddings,
%   one has 
%   \[
%     D\MOD(\GR_G) \simeq \LIM_i D\MOD(X_i)
%   \]
%   where the limit is taken along !-pullbacks of $D$-modules.
%   \footnote{
%     See \cite[Section 2]{BDSeminar3}
%   }
%   \item One can find a presentation $\GR_G = \COLIM_{i} X_i$
%   of the affine Grassmannian as the colimit of
%   projective varieties $X_i$ along closed embeddings such that
%   each $X_i$ is $L^+G$ invariant and the action of $L^+G$
%   factors through an algebraic quotient $G_i$.
%   One can choose the algebraic quotients such that
%   for $X_i \subs X_j$, $G_j \twoheadrightarrow G_i$.
%   Finally, one defines 
%   \[
%     D\MOD(\GR_G)^{L^+ G} := 
%     \COLIM_{i} D\MOD(X_i)^{G_i}
%   \]
%   where $D\MOD(X_i)^{G_i}$ is something I will explain now.
% \end{enumerate}
Provided that we know how to deal with $D$-modules on
infinite type schemes like $\GR_G, LG$ and
their quotients nice enough actions by pro-algebraic groups like $L^+G$,
we can now define a monoidal structure
on $\SPH_G$ via convolution.
\begin{dfn}[Convolution product]
  
  The multiplication on the loop group induces a diagram
  involving $\HECKE^\LOC$
  \begin{cd}
    {L^+G \backslash LG / L^+G} & {L^+G \backslash LG \times^{L^+G} LG / L^+G} & {L^+G \backslash LG / L^+G} \\
    & {L^+G \backslash LG / L^+G}
    \arrow["{p_0}"', from=1-2, to=1-1]
    \arrow["{p_1}"', from=1-2, to=2-2]
    \arrow["m", from=1-2, to=1-3]
  \end{cd}
  The convolution product is defined by the following formula 
  \[
  \SPH_G \times \SPH_G \to \SPH_G , 
  (\cF_0 , \cF_1) \mapsto \cF * \cG := 
  m_!(p_0^! \cF_0 \otimes p_1^! \cF_1) 
  \]
  and defines a monoidal structure on $\SPH_G$.
\end{dfn}

\section{$D$-modules on quotient stacks}

Let $X$ be a variety over $k$ and $G$ an algebraic group
acting on $X$.
With this, one can form the truncated Cech nerve of $X \to X / G$.
\begin{cd}
  {X \times_{X / G} X \times_{X / G} X} & 
  {X \times_{X / G} X} & {X}
  \arrow["{p_0}"{description}, shift left=2, from=1-2, to=1-3]
  \arrow["{p_{12}}"{description}, from=1-1, to=1-2]
  \arrow["{p_{02}}"{description}, shift right=4, from=1-1, to=1-2]
  \arrow["{p_{01}}"{description}, shift left=4, from=1-1, to=1-2]
  \arrow["{p_1}"{description}, shift right=2, from=1-2, to=1-3]
\end{cd}
where we have the projections 
\begin{itemize}
  \item $p_i : X \times_{X/G} X \to X, (x_0 , x_1) \mapsto x_i$
  \item $p_{ij} : X \times_{X / G} X \times_{X / G} X \to X \times_{X/G} X, 
  (x_0 , x_1, x_2) \mapsto (x_i , x_j)$
\end{itemize}
One can show that this is precisely the usual action groupoid :
\begin{cd}
  {G \times G \times X} & {G \times X} & {X}
  \arrow[shift left=2, from=1-2, to=1-3]
  \arrow[from=1-1, to=1-2]
  \arrow[shift right=4, from=1-1, to=1-2]
  \arrow[shift left=4, from=1-1, to=1-2]
  \arrow[shift right=2, from=1-2, to=1-3]
\end{cd}
For an affine $S$,
an $S$-point of $X \times_{X / G} X$ is a triple
$(x_0 , x_1 , \phi)$ where $x_i \in X(S)$ and 
\begin{cd}
  {S \times G} & X \\
	S & {S \times G}
	\arrow[from=1-1, to=2-1]
	\arrow[from=2-2, to=2-1]
	\arrow[from=1-1, to=1-2]
	\arrow[from=2-2, to=1-2]
	\arrow["\phi"{description}, from=1-1, to=2-2]
\end{cd}
where $\phi$ is $G$-equivariant.
The data of $\phi$ is equivalent to $g \in G(S)$ such that
$x_1 = g \cdot x_0$.
% Here is a trick to remember what these maps do : 
% \begin{enumerate}
%   \item For an affine $S$, 
%   an $S$-point of $G \times X$ is a pair $(g , x) \in G(S) \times X(S)$.
%   Draw a ``1-simplex''
%   \begin{cd}
%     x & {g \cdot x}
%     \arrow["g", squiggly, from=1-1, to=1-2]
%   \end{cd}
%   The map $p_i : G \times X \to X$ takes the $i$-th vertex of the 1-simplex.
%   \item For an affine $S$,
%   an $S$-point of $G \times G \times X$ is a triple
%   $(g_2 , g_1 , x) \in G(S) \times G(S) \times X(S)$.
%   Draw a ``2-simplex''
%   \begin{cd}
%     x & {g_1 \cdot x} \\
%     & {g_2 \cdot (g_1 \cdot x)}
%     \arrow["{g_1}", squiggly, from=1-1, to=1-2]
%     \arrow["{g_2}", squiggly, from=1-2, to=2-2]
%     \arrow["{g_2 g_1}"', squiggly, from=1-1, to=2-2]
%   \end{cd}
%   The map $p_{ij}$ picks out the pair of $i$-th vertex and
%   the edge connecting the $i$-th vertex to the $j$-th vertex.
% \end{enumerate}

\begin{dfn}[$G$-equivariance]
  
  The category $(\QCOH X)^G$ of quasi-coherent sheaves on $X$
  equipped with $G$-equivariance is defined as follows :
  \begin{itemize}
    \item An object of $(\QCOH X)^G$ is 
    a quasi-coherent sheaf $\cF \in \QCOH X$
    equipped with a 
    $\phi : p_0^* \cF \cong p_1^* \cF$ 
    in $\QCOH(X \times_{X / G} X)$ such that
    \begin{enumerate}
      \item $\phi = \id{}$ when restricted to the diagonal section 
      $\De : X \to X \times_{X / G} X$
      \item the following diagram commute in 
      $\QCOH(X \times_{X / G} X \times_{X / G} X )$ :
      \begin{cd}
        & {p_{01}^*p_1^* \mathcal{F}} \\
        {p_{01}^*p_0^*\mathcal{F}} && {p_{12}^*p_0^*\mathcal{F}} \\
        {p_{02}^*p_0^*\mathcal{F}} && {p_{12}^*p_1^*\mathcal{F}} \\
        & {p_{02}^*p_1^* \mathcal{F}}
        \arrow["{p_{01}^*(\phi)}", from=2-1, to=1-2]
        \arrow["{p_{12}^*(\phi)}", from=2-3, to=3-3]
        \arrow["{p_{02}^*(\phi)}"', from=3-1, to=4-2]
        \arrow["\sim"', from=2-1, to=3-1]
        \arrow["\sim", from=1-2, to=2-3]
        \arrow["\sim", from=3-3, to=4-2]
      \end{cd}
    \end{enumerate}
    \item a morphism $\eta : (\cF , \phi) \to (\cG , \psi)$ is
    a morphism $\eta : \cF \to \cG$ in $\QCOH X$ such that
    the following commutes : 
    \begin{cd}
      {p_0^* \cF} & {p_1^* \cF} \\
      {p_0^* \cG} & {p_1^* \cG}
      \arrow["{p_0^*(\eta)}"', from=1-1, to=2-1]
      \arrow["{p_1^*(\eta)}", from=1-2, to=2-2]
      \arrow["\psi"', from=2-1, to=2-2]
      \arrow["\phi", from=1-1, to=1-2]
    \end{cd}
  \end{itemize}
  A weakly $G$-equivariant $D$-module on $X$
  is a $D$-module on $X$ equipped with $G$-equivariance structure $\phi$
  which is a morphism of $\cO_G \boxtimes D_X$-modules,
  where we used $G \times X \simeq X \times_{X / G} X$.

  A strongly $G$-equivariant $D$-module on $X$
  is the same except we require $\phi$ is be a
  morphism of $D_{G \times X}$-modules.
\end{dfn}

Using the equivalence of $D$-modules and crystals : 
\[
  D\MOD(X) \simeq \QCOH X_\DR
\]
the above definitions can be rephrased as follows : 
\begin{enumerate}
  \item The action of $G$ on $X$ induces an action of $G$ on $X_\DR$.
  Thus, we have an action groupoid for $X_\DR$ : 
  \begin{cd}
    {G \times G \times X_\DR} & {G \times X_\DR} & {X_\DR}
    \arrow[shift left=2, from=1-2, to=1-3]
    \arrow[from=1-1, to=1-2]
    \arrow[shift right=4, from=1-1, to=1-2]
    \arrow[shift left=4, from=1-1, to=1-2]
    \arrow[shift right=2, from=1-2, to=1-3]
  \end{cd}
  Then a weakly $G$-equivariant $D$-module on $X$
  is equivalently a $G$-equivariant quasi-coherent sheaf on $X_\DR$.
  \item Taking de Rham spaces commutes with limits of presheaves.
  \footnote{
    In fact, it commutes with colimits of presheaves too.
  }
  So $(G \times X)_\DR = G_\DR \times X_\DR$.
  Then the action of $G$ on $X$ induces an action of
  $G_\DR$ on $X_\DR$, giving us the action groupoid 
  \begin{cd}
    {G_\DR \times G_\DR \times X_\DR} & {G_\DR \times X_\DR} & {X_\DR}
    \arrow[shift left=2, from=1-2, to=1-3]
    \arrow[from=1-1, to=1-2]
    \arrow[shift right=4, from=1-1, to=1-2]
    \arrow[shift left=4, from=1-1, to=1-2]
    \arrow[shift right=2, from=1-2, to=1-3]
  \end{cd}
  A strongly $G$-equivariant $D$-module on $X$
  is thus equivalently a $G_\DR$-equivariant quasi-coherent sheaf on $X_\DR$.
\end{enumerate}

We have the following.
\begin{prop}
  
  There is an equivalence of categories : 
  \[
    (\QCOH X_\DR)^{G_\DR} \simeq D\MOD(X / G)
  \]
  i.e. strongly $G$-equivariant $D$-modules on $X$
  are equivalent to $D$-modules on $X / G$.
\end{prop}
\begin{proof}(Sketch)
  
  First, we define the functor $(\QCOH X_\DR)^{G_\DR} \to D\MOD(X / G)$.
  Recall for a stack $\mathscr{X}$, 
  $\mathcal{D}_{\mathscr{X}}$-modules on it are defined locally 
  in the smooth topology $\mathscr{X}_{sm}$.
  The objects of $\mathscr{X}_{sm}$ are pairs $(S,\pi_S)$ 
  where $S$ is a scheme and 
  $\pi_S : S \to \mathscr{X}$ is a smooth $1$-morphism. 
  \cite[Section 1.1.2]{BD}
  The morphisms between pairs $(S,\pi_S)$ and 
  $(S^\prime, \pi_{S^\prime})$ are pairs $(\varphi,\alpha)$ 
  where $\varphi: S \to S^\prime$ is a smooth morphism 
  and $\alpha: \pi_S \xrightarrow[]{\sim} \pi_{S^\prime} \varphi$ is 
  a $2$-morphism, i.e., it satisfies some kind of cocycle condition. 
  Suppose we now have a strongly equivariant $D$-module $M$ on $X$.
  Goal : for each $(S,\pi_S) \in (X / G)_{sm}$, 
  we need to give a $\mathcal{D}_S$-module $M_S$, 
  and, given a morphism 
  $(\varphi, \alpha):(S,\pi_S) \to (S^\prime,\pi_{S^\prime})$ 
  we need to give an isomorphism of 
  $\mathcal{D}_{S}$-modules 
  $\beta: \varphi^* M_{S^\prime} \xrightarrow[]{\sim} M_{S}$ 
  satisfying a cocycle condition. 
  The data of $(S , \pi_S)$ is equivalent to 
  giving a principal $G$-bundle $p:P \to S$ together with 
  a smooth $G$-equivariant morphism $\gamma: P \to X$. 
  \begin{equation}
  \begin{tikzcd}
    P & X \\
    S & {X/G}
    \arrow["\gamma", from=1-1, to=1-2]
    \arrow["p"', from=1-1, to=2-1]
    \arrow[from=2-1, to=2-2]
    \arrow[from=1-2, to=2-2]
    \arrow["\lrcorner"{anchor=center, pos=0.125}, draw=none, from=1-1, to=2-2]
  \end{tikzcd}
  \end{equation}
  The point now is that $M$ pullback to a strongly equivariant 
  $D$-module $N$ on $P$.
  The $G$-invariant sections $N^G$ of $N$ then 
  determine a $\mathcal{D}_S$-module $M_S$ on $S$. 
  This is because $\mathcal{D}_S$ embeds into $(\mathcal{D}_P)^G$: 
  if $P = G \times S$ is trivial then 
  $\mathcal{D}_S \hookrightarrow (\mathcal{D}_{G \times S})^G$, 
  and one can use a fpqc-local triviality of $G$-bundles 
  to reduce to this case. 
  One can check that this procedure produces 
  a well-defined $D$-module on $X / G$.

  For a quasi-inverse functor,
  note that amongst $(X / G)_{\text{sm}}$ are
  the following fiber squares
  \begin{cd}
    {X \times_{X / G} X} & X & \simeq & {G \times X} & X \\
    X & {X / G} && X & {X / G}
    \arrow[from=1-1, to=2-1]
    \arrow[from=2-1, to=2-2]
    \arrow[from=1-1, to=1-2]
    \arrow[from=1-2, to=2-2]
    \arrow["\lrcorner"{anchor=center, pos=0.125}, draw=none, from=1-1, to=2-2]
    \arrow["{g , x \mapsto x}"', from=1-4, to=2-4]
    \arrow[from=2-4, to=2-5]
    \arrow[from=1-5, to=2-5]
    \arrow["{\text{act}}", from=1-4, to=1-5]
    \arrow["\lrcorner"{anchor=center, pos=0.125}, draw=none, from=1-4, to=2-5]
  \end{cd}
  \begin{cd}
    {X \times_{X / G} X \times_{X / G} X } & X & \simeq & {G \times G \times X} & X \\
    {X \times_{X / G} X} & {X / G} && {G \times X} & {X / G}
    \arrow[from=1-5, to=2-5]
    \arrow[from=2-4, to=2-5]
    \arrow[from=1-4, to=2-4]
    \arrow[from=1-4, to=1-5]
    \arrow["\lrcorner"{anchor=center, pos=0.125}, draw=none, from=1-4, to=2-5]
    \arrow[from=1-2, to=2-2]
    \arrow[from=1-1, to=1-2]
    \arrow["\lrcorner"{anchor=center, pos=0.125}, draw=none, from=1-1, to=2-2]
    \arrow[from=2-1, to=2-2]
    \arrow[from=1-1, to=2-1]
  \end{cd}
  There are three of these second kind
  corresponding to the three projections $p_{01}, p_{12} , p_{02}$
  of the action groupoid of $G$ on $X$.
  \footnote{
    The action map $G \times X \to X$ is smooth because it is the 
    composition of $G \times X \simeq G \times X , (g , x) \mapsto (g , g x)$
    with the second projection.
  }
  Now given a $D$-module $\mathscr{N}$ on $X / G$,
  the cartesian square of the first kind
  gives a $D$-module $N$ on $X$.
  There are morphisms between the cartesian square of the first kind
  and the cartesian squares of the second kind
  in $(X /G)_{\mathrm{sm}}$.
  These produce the transition map for a strong $G$-equivariance structure
  on $N$ and 
  the cocycle condition for $\mathscr{N}$ gives
  the cocycle condition for the the strong $G$-equivariance.

\end{proof}

\section{Relativising to the entire curve}

Back to Hecke actions.
The previous discussion of $\SPH_G$ does not involve the curve $X$.
Given a closed point $x$ of the curve $X$
and a choice $k\bbrkt{t} \simeq \cO^\wedge_x$,
we can replace the disk and punctured disk around $0$ in $\bA^1$
by the disk and punctured disk around $x$ in $X$.
(If need more material,
can define $D_{x_S}, D_{x_S}^\circ$.)
In other words, we have isomorphisms : 
\[
  LG \simeq L_x G \,\,\,\,\,\,
  L^+G \simeq L^+_x G
\]
where $L_x G (S) = G(D_{x_S}^\circ)$
and $L_x^+ G(S) =  G(D_{x_S})$.
These isomorphisms induce 
\[
  \SPH_G \simeq \SPH_{G,x}
\]
where the latter is the category of $D$-modules on 
$L^+_x G \backslash (L_x G / L_x^+ G)$.
From the talk in week 1, 
we can make an action of $\SPH_{G , x}$ on $D\MOD(\BUN_G)$
by considering the $L_x^+G$-bundle $\BUN_{G , x}^{\text{lvl}} \to \BUN_G$
and the regluing action of $L_x G$ on $\BUN_{G , x}^{\text{lvl}}$.
I now justify my claim in week 1 that there is a way of doing all of this
in families above $X$.

The trick is to consider the ``bundle of uniformisers'' : 
\[
  \widehat{X}^+(S) := \set{(x , \al) \st x \in X(S) , 
  \al : D_{x_S} \simeq D_S \text{ preserving $S$}}
\]
Examples of points of this is given by 
\cite[\href{https://stacks.math.columbia.edu/tag/05D5}{Tag 05D5}]{stacks-project}.
The forgetful map $\widehat{X}^+ \to X$ is a bundle for
the pro-algebraic group $\AUT^+ D$ of automorphisms of $D$
preserving $0$.
There is an action of $\AUT^+ D$ on $LG , L^+ G , \GR_G$
then the relativisation precedure is twisting these
using $\widehat{X}^+$.
\begin{align*}
  L_X^+ G &:= \widehat{X}^+ \times^{\AUT^+ D} L^+ G \\
  L_X G &:= \widehat{X}^+ \times^{\AUT^+ D} L G \\
  \GR_{G , X} &:= \widehat{X}^+ \times^{\AUT^+ D} \GR_G
\end{align*}
The final object here is called the \emph{Beilinson--Drinfeld Grassmannian}
and it will play a crucial role in proving
that the monoidal structure on $\SPH_G$ is symmetric,
i.e. the categorification of the fact that
convolution product on the spherical Hecke algebra is commutative.



\printbibliography

\end{document}